
% Chapter 4
\chapter*{Conclusion générale}

\label{Chapter4} % Change X to a consecutive number; for referencing this chapter elsewhere, use \ref{ChapterX}

\lhead{\emph{Introduction générale}} % Change X to a consecutive number; this is for the header on each page - perhaps a shortened title

%------------------------------------------------------------------------------
%	INTRODUCTION GENERALE
%----------------------------------------------------------------------------------------

\paragraph*{}
Nous avons parcouru, tout au long de ce mémoire, les différentes étapes de la segmentation d'image basées sur la méthode de reconnaissance de texture, et nous pouvons dire que ce travail démontre la puissance de cette dernière.

\paragraph*{}
Nous avons étudié les aspects statistiques de la caractérisation des textures, et plus précisement, la caractéristique LBP qui nous a offert un outils de reconaissance très puissant. Nous avons aussi étudié la méthode de segmentation dynamique qui permet de détecter une texture donnée, dans une image donnée, 
Notre travail consistait à élaborer une méthode de généralisation pour la détection automatique de toutes les textures présentes dans une image, sans avoir, au préalable, aucune information à part l'image elle même.
Les résultats obtenu restent très prometteurs et motivants pour accorder à cette étude plus de temps et d'efforts. 

\paragraph*{}
Bien que certaines contraintes ont du être respectés, il faut savoir que ce travail est une première étape d'application de cette méthode. Néomoin, nous aurions souhaité bénificier de plus de temps pour mettre en application nos idée qui consistent à, premièrement, lier la reconnaissance et la segmentation de texture à des descriptions dans un format compréhensible par une machine. Celle ci, pourrait très bien comprendre si elle est en train de survoler une zone forestière ou urbaine. 

\paragraph*{}
Aussi, pour trouver un accord entre les différentes formes géométriques pour la détection des textures, nous pensons on continuer avec une reconnaissance de formes régulières. Ce travail peut-etre enrichi en implémentant l'algorithme avec des formes hexagonales. 
Ces formes représentent, théoriquement, un très bon compromis entre les différentes autres formes. Leur nature permet d'englober uniformément une surface entière tout en évitant les chevauchements (contrairement aux cercles, par exemple). Son avantage par rapport aux formes carrées et aux triangles équilatérales est le fait qu'un hexagone couvre bien plus efficacement une surface ronde. En effet, le rapport entre la surface d'un triangle équilatéral et un cercle est de 17.77\%. Celle d'un carré est de 63.7\%. En fin, celle d'un hexagone n'est pas moins de 83\%.

\paragraph*{}
Nous avons décidé de continuer personellement ce travail afin d'aboutire à d'autres résultats souhaités.

%----------------------------------------------------------------------------------------


