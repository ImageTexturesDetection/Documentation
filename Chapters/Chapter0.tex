
% Chapter 0
\chapter*{Introduction générale}

\label{Chapter0} % Change X to a consecutive number; for referencing this chapter elsewhere, use \ref{ChapterX}

\lhead{\emph{Introduction générale}} % Change X to a consecutive number; this is for the header on each page - perhaps a  shortened title



%------------------------------------------------------------------------------
%	INTRODUCTION GENERALE
%----------------------------------------------------------------------------------------

\paragraph*{}
Il est bien simple et évident pour nous, humains, de reconnaître ce que nous voyons instantanément.  Cependant, ce n'est pas aussi simple pour un ordinateur. Étant donné la puissance et la compléxité des traitements effectués par notre cerveau, il nous est très difficile de les analyser, et d'autant plus à les reproduire. 

\paragraph*{}
Si on prend en exemple la reconnaissance et la segmentation des textures qui sont aussi une partie du domaine du traitement d’images, nous nous retrouvons dans le paradoxe de l’étude d’une entité non défini mais parfaitement reconnaissable. Le fait de reconnaître un ciel bleu et le distinguer d’une rivière ou des roches tout au tour est élémentaire pour le cerveau humain et utilisé instinctivement par ce dernier. On peut aussi très facilement distinguer une table en plastique d’une table en bois. Cependant, la simple définition du  terme «\textit{texture}» reste ambiguë. 

\paragraph*{}
Comment pouvons nous donc traiter une entité indéfinie?

\paragraph*{}
La réponse se trouve dans les études statistiques sur les différentes relations qui existent entre les pixels qui forment une image.
Les chercheurs ont pu trouver différentes formules et caractéristiques afin de définir et segmenter les textures. La plupart du temps, l'étude s'intéresse surtout à la détection de motifs réguliers formant la texture, et cela en comparant chaque pixel avec la disposition de son entourage.

\paragraph*{}
La plupart des travaux fait dans ce domaine jusque là, visent à reconnaître une texture dans une image en ayant initialement une partie de cette première comme référence. Ceci est employé dans la recherche de motif déjà connu.

\paragraph*{}
Notre travail consiste à proposer une généralisation de ces méthodes afin de pouvoir l'implémenter dans des systèmes autonomes. Ces derniers doivent être capables d'extraire automatiquement les informations d'une image digitale. En d'autres termes, ils doivent pouvoir segmenter les différentes textures qu'on leur présente.\\


\paragraph*{}
Notre mémoire est organisé en trois chapitre. Nous présenterons dans le premier chapitre des généralités, des définitions et des notions de bases sur les méthodes utilisées dans notre travail. Dans le second chapitre nous introduirons en détails le principe de l'approche utilisée pour la segmentation des textures. Le troisième sera consacré à la présentation de notre application ainsi que les testes réalisés sur des images générées aléatoirement.

%----------------------------------------------------------------------------------------