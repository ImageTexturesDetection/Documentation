\renewcommand{\headrulewidth}{0pt}
\setcode{utf8}


\begin{center}
\begin{Huge}
\textbf{ABSTRACT}\\

\end{Huge}
\end{center}

This study is an initial attempt to generalize the Texture Segmentation and Matching using LBP Operator method. \\

This thesis first examines the principle of a segmentation and matching algorithm which is to detect a given textures over a given image. 
Then we'll describe our implementation of that automatic detection algorithm that detects all textures in an image with no other information given but the targeted image.
Finally, the method is applied on randomly generated images from a database of over 30 textures.\\

On the basis of the results of this research, it can be concluded that texture recognition methods are indeed very powerful and deserve spending more efforts and time on.\\



\begin{center}
\begin{Huge}
\textbf{RESUMÉ}\\

\end{Huge}
\end{center}


\begin{center}
\begin{Huge}
\<ملخص>\\
\end{Huge}
\end{center}


 \<
 هذه الدراسة محاولة أولية لتعميم نظام تقسيم ومطابقة الانسجة باستخدام طريقة  .
  سندرس في بداية هذا البحث مبدأ خوارزمية تقسيم ومطابقة الانسجة اللذي  يستعمل للتعرف على نسيج معين داخل صورة معينة.
  و من ثم سنقوم وصف تطبيقنا لتلك الخوارزمية الكشف الأوتوماتيكي اللتي تسمح بالكشف عن كافة انسجت صورت ما، و ذلك بدون الاعتماد على اي معلومة اولية.
 أخيرا، سنقوم تطبيق الطريقة على صور مركبة  بشكل عشوائي من قاعدة بيانات لأكثر من 30 نسيج.
على أساس نتائج هذا البحث، فإنه يمكن استنتاج أن وسائل التعرف على الانسجة هي في الواقع قوية جدا وتستحق إنفاق المزيد من الجهود والوقت في دراستها.
>
